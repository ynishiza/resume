\section{Work Experience}
\cventry
  {Software Developer} % Job title
  {Exaptive Inc.} % Organization
  {Kobe, Hyogo, Japan} % Location
  {June 2014 - August 2022} % Date(s)
  {
    \begin{itemize} % Description(s) of tasks/responsibilities
      \liststyle
      \item {\tgproject{Cognitive city} a general knowledge base
        % - skills: \rfnodejs, distributed systems, \rfmongodb, \rfneofj, \rfredis, \rfreact, \rfaws, \rfterraform, \rfansible, \rfjenkins, \rfnewrelic, Elasticsearch \\
        \tgsubitemlabel{Primary designer and developer of the application backend} 
          the core of the backend was a {\it multi-model database} (\rfmongodb, \rfneofj, \rfelastic) and a {\it REST API}, implemented with a {\it distributed architecture} written in \rfnodejs. 
          Designed and implemented the main system architecture, data structures, and APIs. 
          Implemented various features to scale the system in correspondence with usage. 
          Some topics addressed were:
          data consistency and synchronization,
          data migration,
          parallelization,
          crash recovery,
          error handling and error code schemes,
          access control system (RBAC, ABAC).
          Also managed {\it site reliability} features, such as structured logging and logging practices, and setting up events, monitors and alerts in \rfnewrelic (initially also \rfsumologic).
        \tgsubitemlabel{Testing} introduced rigorous testing practices to ensure that every feature and changes are tested thoroughly, including security tests. 
        Implemented custom testing tools and frameworks to facilitate and thereby encourage testing.
        1800 tests at time of leaving. 
        Also helped redesign a testing framework in \rfcypress for the frontend team where testing was not as effective.  
        \tgsubitemlabel{Other} assisted in frontend development and DevOps as necessary. 
        \tgskills (Backend) \rfnodejs, \rfmongodb, \rfneofj, \rfredis,  \rfbash, \rfdocker, 
        (DevOps) \rfaws, \rfterraform, \rfansible, \rfjenkins, \rfnewrelic, \rfsumologic,
        (Frontend) \rfreact, \rfcypress
      }
      \mynotes{
Project: Cognitive City
Description: 

Primary designer and developer of application backend.
Backend: a {\it multi-model database} (\rfmongodb, \rfneofj, \rfelastic) with a {\it REST API}
Implementation: a {\it distributed architecture} in \rfnodejs. 

Designed and implemented the main system architecture, data structures, and APIs. 

Features:
- safe data migration: a migration system consisting of a migration history 
- data consistency: with write lock
  concepts: message queue
- system architecture: separation of REST API layer from the main application process layer
  concepts: message queue, worker processes
- synchronization: consistent (i.e. order respecting) synchronization of data across the multi database models
  concepts: message queue, worker processes
- error handling: error handling and error code scheme
- fault tolerance: crash recovery
- security: access control
  role based access control (RBAC) as basis
  attribute based access control (ABAC) for fine-grained control e.g. limit control by element owner

Testing:
Introduced rigorous testing practices to ensure that every feature and changes are tested thoroughly, including security tests (e.g. access control). 
1800 tests at time of leaving. 
Tests were robust enough that we felt comfortable refactoring any part of the code base and not fear losing and breaking a feature or introducing a bug in the process.
Implemented custom testing tools and frameworks to facilitate and thereby encourage testing
- an assertion library for asserting on HTTP requests easily
- a framework for declaratively defining access control tests based on user policy configuration.
- tool for easily managing common artifacts in a test setup
  e.g. each test needs a user
  A tool for creating a new user, setting user policies, logging the user in, etc.
  e.g. tests often need to interact with data in the knowledge graph
  A tool for creating new data in the knowledge graph, modifying, deleting, querying, etc


Site reliability:
- logging: defined logging practice 
  - definition of each log level
  - what should be in a message
  Implemented tool to support this practice.
- logging implemented structured logging to facilitate parsing logs
- logging: defined custom New Relic events for tracking uniform processes
  e.g. event to track the start and end/error of a worker process
- monitoring: setup monitors and alerts in New Relic



data consistency and synchronization,
data migration,
parallelization,
crash recovery,
error handling and error code schemes,
access control system (RBAC, ABAC).
      }
      \item {\tgproject{Exaptive Studio} A Low-Code platform, featuring not only client-side but also server-side components in \rfpython 2 and \rfrlang 
        \tgsubitemlabel{Primary developer of components and layout engine}: 
        \tgskills \rfjavascript, \rfpython 2/3, \rfrlang, \rfmysql, \rfredis, \rfdocker, networks (TCP/IP, websockets), \rfaws, \rfansible
        - primary skills: Node.js, distributed systems, Python, R\\
          secondary skills: AWS, Ansible \\
        - DOM, layout
        - interface
        - domain 
        - protocol
      }
    \end{itemize}
  }

\cventry
  {Software Developer} % Job title
  {Skybound Software} % Organization
  {Waterloo, ON, Canada} % Location
  {2010-2011} % Date(s)
  {
    \begin{itemize} % Description(s) of tasks/responsibilities
      \item {skills: \rfcsharp, \rftypescript, \rfhtml/\rfcss \rfbash \ scripting, \rfaws}
      \item{\tgproject{Glide, a WYSIWYG webpage editor}
      \tgskills \rfcsharp, \rftypescript, \rfhtml/\rfcss, \rfbash, \rfaws
      }
      \item{\tgproject{Selective API, a reactive programming data structure with a LINQ interface}
      this was before reactive programming started becoming popular, such as \rfreact.
      \tgsubitem designed and implemented a complete set of automated tests for the data structure, 
      and fixed any issues found in the process, including bugs and performance issues. 
      Also performed additional performance analysis and optimizations.
      \tgsubitem documented the specifications of the data structure by establishing precise definitions and rigorously ensuring correctness of its functions.
      \tgskills \rfcsharp, \rflatex
      }
    \end{itemize}
  }

\cventry
  {Computational Research Assistant (Internship)} % Job title
  {CANMET Materials Technology Laboratory (NRCan)} % Organization
  {Ottawa, ON, Canada} % Location
  {2008 - 2009} % Date(s)
  {
    \begin{itemize} % Description(s) of tasks/responsibilities
      \item {a}
    \end{itemize}
  }

\cventry
  {Research Assistant (Internship)} % Job title
  {Center of Atmospheric Sciences (University of Waterloo)} % Organization
  {Jan 2008 - Apr 2008} % Location
  {} % Date(s)
  {
    \begin{itemize} % Description(s) of tasks/responsibilities
      \item {a}
    \end{itemize}
  }

\cventry
  {Quality Analyst (Internship)} % Job title
  {Quest Software} % Organization
  {Toronto, ON, Canada} % Location
  {May 2007 - Aug 2007} % Date(s)
  {
    \begin{itemize} % Description(s) of tasks/responsibilities
      \item {a}
    \end{itemize}
  }
\cventry
  {IT Assistant (Internship)} % Job title
  {Canadian Academy} % Organization
  {Kobe, Hyogo, Japan} % Date(s)
  {Sep 2006 - Dec 2006} % Location
  {
    \begin{itemize} % Description(s) of tasks/responsibilities
      \item {a}
    \end{itemize}
  }
