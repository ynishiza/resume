\section{Work Experience}

\cventry
  {Software Developer} % Job title
  {Exaptive} % Organization
  {OKC, Oklahoma, USA (remote from Ashiya, Japan)} % Location
  {June 2014 - August 2022} % Date(s)
  {
    \begin{itemize} % Description(s) of tasks/responsibilities
      \liststyle
      \item {\tgproject{Cognitive city (2018-2022)} a generic graph-model knowledge base, branded as a ``social network of ideas''.
        I was the primary designer and developer of the application backend.
        In addition to the core knowledge base, the backend supported various other services, 
        such as user management and authentication, API key, domain customization, and data exploration and analysis tools.
        A distributed system written in Node.js backed by MongoDB, Neo4j and Elasticsearch, through which I studied and employed various principles such as 
        authentication+authorization, 
        data syncing+consistency, 
        data migration management,
        process parallelization+coordination, 
        fault tolerance, 
        access control, 
        process logging + monitoring, 
        etc.
        \tgsubitem test suite: one of my favorite accomplishments. Introduced rigorous testing practice, thoroughly testing every change and feature, including security. 
        1800 tests at time of leaving. 
        Its robustness allowed refactoring and refining the code to also become regular practice, thus constantly retaining high code quality. 
        Also tried to share this success on the frontend which was yet experiencing challenges, by designing and implementing a test suite in Cypress.
        \tgsubitem assisted with frontend and DevOps as necessary. 
        \tgskills (Backend) \rfnodejs, \rfmongodb, \rfneofj, \rfredis, \rfelastic, \rfbash, \rfdocker, 
        (DevOps) \rfjenkins, \rfaws, \rfansible, \rfterraform, \rfnewrelic, \rfsumologic,
        (Frontend) \rftypescript, \rfreact, \rfcypress
      }

      \item {
        \tgproject{Exaptive Studio (2014-2018)} a low-code webplatform, featuring not only components client-side in JavaScript but also server-side in Python 2 (+Python 3 unofficially) and R, and UI in HTML/CSS.
        I was the designer and developer for many of the core parts, such as 
        the user-facing component interface and library package for all programming languages, 
        UI layout engine with a virtual DOM,
        component resource management and caching system, including JS dependency conflict handling (e.g use both D3 v6 and v7),
        and 
        server-side component execution system including an RPC and a multipart protocol over TCP/IP.
        \tgskills  \rfjavascript, \rfpython\ 2/3, \rfclang, \rfphp, \rfmysql, \rfredis, \rfdocker, TCP/IP, websockets, \rfhtml/\rfcss, \rfaws, \rfdthree
      }
    \end{itemize}
  }

\cventry
  {Software Developer} % Job title
  {Skybound Software} % Organization
  {Waterloo, ON, Canada} % Location
  {2011-2013} % Date(s)
  {
    \begin{itemize} % Description(s) of tasks/responsibilities
      \liststyle
      \item{\tgproject{a WYSIWYG webpage editor} initially in C\#, then redesigned and implemented in TypeScript. 
      My main contribution was redesigning and implementing a more efficient layout engine, which converts the user design into corresponding HTML/CSS.
      Also assisted with frontend and DevOps as needed, such as writing shell scripts to deploy updates to AWS.
      \tgskills VMWare, Visual Studio, \rfcsharp, \rftypescript, \rfhtml/\rfcss, \rfmercurial, \rfbash, \rfaws
      }

      \item{\tgproject{Selective API, a reactive programming data structure with LINQ interface}
      this was before reactive programming became main stream, such as with React.
      My main task was designing and implementing a complete set of automated tests for the data structure, and fixing any issues found in the process, including performance issues. \\
      Also documented full specifications of the data structure, establishing logically precise definitions and verifying functional correctness.
      \tgskills VMWare, Visual Studio, \rfcsharp, \rflatex, \rfmercurial

      }
    \end{itemize}
  }

\cventry
  {Computational Research Assistant (Co-op)} % Job title
  {CANMET Materials Technology Laboratory (NRCan)} % Organization
  {Ottawa, ON, Canada} % Location
  {2008 - 2009} % chktex 8
  {
    \begin{itemize} % Description(s) of tasks/responsibilities
      \liststyle
      \item{
        Researched microscopic point-defects in iron due to irradiation in nuclear reactors using simulations.}
      \item{Operated \href{https://www.lammps.org/}{LAMMPS}, a molecular dynamics simulator, on a public remote computer cluster. 
        Co-worked on a shell script suite to automate simulation run management, including scheduling and result retrieval.
      }
    \item{
      \tgproject{Point-defect analysis tool} designed and implemented a Fortran 95 tool to analyze point-defects in simulation runs by employing geometric and statistical tools. 
      Performed point-defect identification and classification, and cluster identification.
      Devised a new way to dynamically select a key threshold in place of the traditional manual value by statistical modelling.
\tgskills \rffortran, \rfclang
    }
    \end{itemize}
  }



\cventry
  {Research Assistant (Co-op)} % Job title
  {Center of Atmospheric Sciences (University of Waterloo)} % Organization
  {Waterloo, ON, Canada} % Date(s)
  {Jan 2008 - Apr 2008} % Location
  {
    \begin{itemize} % Description(s) of tasks/responsibilities
      \liststyle
      \item {Helped reproduce snowflake-like ice particles with clear crystalline shapes for the first time in the lab's flowtube. The key was adding more moisture.}
      \item {\tgproject{3D shape identification} designed and implemented a C/C++ program to identify the 3D shape of ice particles from images taken in the flowtube. 
      The program applied the theory of discrete Fourier shape descriptors.
      I reduced a core integral equation used at the lab into a simpler form.
      \tgskills: \rfcpp, \rfmysql}
    \end{itemize}
  }

\cventry
  {Quality Analyst (Co-op)} % Job title
  {Quest Software} % Organization
  {Toronto, ON, Canada} % Location
  {May 2007 - Aug 2007} % Date(s)
  {
    \begin{itemize} % Description(s) of tasks/responsibilities
      \liststyle
      \item {Tested Foglight, a performance monitoring software, on various systems including openSUSE, Red Hat Linux, Solaris, and HP-UX.}
    \end{itemize}
  }
\mynotes{
Automated setups and tests using shell scripts.
}

\cventry
  {IT Assistant (Co-op)} % Job title
  {Canadian Academy} % Organization
  {Kobe, Hyogo, Japan} % Date(s)
  {Sep 2006 - Dec 2006} % Location
  {
    \begin{itemize} % Description(s) of tasks/responsibilities
      \liststyle
      \item {Provided technical assistance to school staff.}
      \item {\tgproject{file administration} deployed \href{http://www.radmind.org/}{Radmind}, an open source remote Unix file system administration software for synchronizing managed files across all school lab computers.}
    \end{itemize}
  }
