\section{Work Experience}
- Deployed Radmind, an open source file management software for synchronizing managed files across all school lab computers.
\cventry
  {Software Developer} % Job title
  {Exaptive Inc.} % Organization
  {Kobe, Hyogo, Japan} % Location
  {June 2014 - August 2022} % Date(s)
  {
    \begin{itemize} % Description(s) of tasks/responsibilities
      \liststyle
      \item {\tgproject{Cognitive city} a general knowledge base
        % - skills: \rfnodejs, distributed systems, \rfmongodb, \rfneofj, \rfredis, \rfreact, \rfaws, \rfterraform, \rfansible, \rfjenkins, \rfnewrelic, Elasticsearch \\
        \tgsubitemlabel{Primary designer and developer of the application backend} 
          the core of the backend was a {\it multi-model database} (\rfmongodb, \rfneofj, \rfelastic) and a {\it REST API}, implemented as a {\it distributed architecture} written in \rfnodejs. 
          Designed and implemented the main system architecture, data structures, and interfaces. 
          Most of my day to day work consisted of fine-tuning and scaling the system to keep up with the usage.
          Some topics addressed were:
          process parallelization and coordination,
          data synchronization and consistency,
          crash recovery,
          error handling and error code schemes,
          access control (RBAC, ABAC).
          Also managed {\it site reliability} features, such as structured logging and logging practices, and setting up events, monitors and alerts in \rfnewrelic (initially also \rfsumologic).
        \tgsubitemlabel{Testing} introduced rigorous testing practices to ensure that every feature and changes are tested thoroughly, including security tests. 
        Implemented custom testing tools and frameworks to facilitate and thereby encourage testing.
        1800 tests at time of leaving. 
        Also helped redesign a testing framework in \rfcypress for the frontend team where testing was not as effective.  
        \tgsubitemlabel{Other} assisted in frontend development and DevOps as necessary. 
        \tgskills (Backend) \rfnodejs, \rfmongodb, \rfneofj, \rfredis,  \rfbash, \rfdocker, 
        (DevOps) \rfaws, \rfterraform, \rfansible, \rfjenkins, \rfnewrelic, \rfsumologic,
        (Frontend) \rfreact, \rfcypress
      }
      \mynotes{
Project: Cognitive City
Description: 

Primary designer and developer of application backend.
Backend: 
- a graph database with a multi-model backing:  \rfmongodb as core database, \rfneofj for graph queries, \rfelastic for text search
- interface: {\it REST API}
- Implementation: \rfnodejs 
- architecture: distributed


Features:
- safe data migration: 
A migration system with a migration history 
Advantage: 
  * retain record of migrations
  * can replay migration on any database
Disadvantage:
  * difficult to see the whole schema at any given moment, as changes are dispersed throughout the migration.
    Can workaround to some extent with schema validation.

- application architecture: 
Layers
* application interface layer: mainly handling HTTP requests
* application process layer: handling heavier processes
  e.g. writing data
  e.g. expensive reads, such as graph queries
* background layer: various background processes
  e.g. syncing multi-model database
  Separation of REST API layer from the main application process layer
concepts: message queue, worker processes


- parallelization and coordination: 
  Parallelized work for performance where possible 
  Write lock for data write coordination. Since REST API was separated and cloned, need write lock to safely coordinate writes. 
  tools: message queue


- data synchronization and consistency: 
  Data was synchronization of across the multi database models
  Synchronization was eventually consistent (weak consistency) i.e. Neo4j and Elasticsearch were eventually consistent.
  Monotonic aka order-preserving i.e. synchronization was performed in the order of writes
  concepts: message queue, worker processes


- error handling and error codes:

  
- fault tolerance: 
  Crash recovery 
  * safely restart process on crash
  * 

- access control
  Role based access control (RBAC) as basis
  Attribute based access control (ABAC) for fine-grained control 
   e.g. only element owner can read private element 
   e.g. only user with a given email domain (e.g. gmail.com) can read element


Testing:
Introduced rigorous testing practices to ensure that every feature and changes are tested thoroughly, including security tests (e.g. access control). 
1800 tests at time of leaving. 
Tests were robust enough that we felt comfortable refactoring any part of the code base and not fear losing and breaking a feature or introducing a bug in the process.
Implemented custom testing tools and frameworks to facilitate and thereby encourage testing
- an assertion library for asserting on HTTP requests easily
- a framework for declaratively defining access control tests based on user policy configuration.
- tool for easily managing common artifacts in a test setup
  e.g. each test needs a user
  A tool for creating a new user, setting user policies, logging the user in, etc.
  e.g. tests often need to interact with data in the knowledge graph
  A tool for creating new data in the knowledge graph, modifying, deleting, querying, etc


Site reliability:
- logging: defined logging practice 
  - definition of each log level
  - what should be in a message
  Implemented tool to support this practice.
- logging implemented structured logging to facilitate parsing logs
- logging: defined custom New Relic events for tracking uniform processes
  e.g. event to track the start and end/error of a worker process
- monitoring: setup monitors and alerts in New Relic



data consistency and synchronization,
data migration,
parallelization,
crash recovery,
error handling and error code schemes,
access control system (RBAC, ABAC).
      }
      \item {\tgproject{Exaptive Studio} A Low-Code platform, featuring not only client-side but also server-side components in \rfpython 2 and \rfrlang 
        \tgsubitemlabel{Primary developer of components and layout engine}: 
        \tgskills \rfjavascript, \rfpython 2/3, \rfrlang, \rfmysql, \rfredis, \rfdocker, networks (TCP/IP, websockets), \rfaws, \rfansible
        - primary skills: Node.js, distributed systems, Python, R\\
          secondary skills: AWS, Ansible \\
        - DOM, layout
        - interface
        - domain 
        - protocol
      }
    \end{itemize}
  }

\cventry
  {Software Developer} % Job title
  {Skybound Software} % Organization
  {Waterloo, ON, Canada} % Location
  {2010-2011} % Date(s)
  {
    \begin{itemize} % Description(s) of tasks/responsibilities
      \liststyle
      \item {skills: \rfcsharp, \rftypescript, \rfhtml/\rfcss \rfbash \ scripting, \rfaws}
      \item{\tgproject{Glide, a WYSIWYG webpage editor}
      \tgskills \rfcsharp, \rftypescript, \rfhtml/\rfcss, \rfbash, \rfaws
      }

      \item{\tgproject{Selective API, a reactive programming data structure with LINQ interface}
      this was before reactive programming, such as \rfreact, started becoming popular. 
      My main task was designing and implementing a complete set of automated tests for the implementing structure, and fixing any issues found in the process, including performance issues. 
      Also documented full specifications of the data structure by establishing logically precise definitions.
      \tgskills Visual Studio, \rfcsharp, \rflatex
      }
    \end{itemize}
\mynotes{
A modern WYSIWYG webpage editor
- Initially implemented in C#, later ported to TypeScript
- redesigned and implemented the HTML layout engine
i.e. a layout engine for converting the user design into corresponding layout in HTML/CSS
- assisted frontend tasks, such as implementing animations
- DevOps: bash scripts to deploy application server to AWS

Skills: C#, TypeScript, HTML/CSS, bash, AWS
Tools: Visual Studio, AWS

Selective API (C#): a reactive programming data structure with LINQ interface
- a reactive data structure (i.e. a chainable data structure that updates each layer when the value of the base changes) before reactive programming became popular
- used for programming the UI e.g. update UI when underlying data changes
- main task: designed and implemented a complete set of unit tests for thoroughly testing the data structure
  Fixed issues found in the process, including performance issues
  Utilized performance analysis tools 
- main task: document full specifications of the Selective API 
  Establish logically precise specification for each feature
  Rigorously ensure correctness of behavior

Skills: C#, LaTeX
Tools: Visual Studio, AWS
}
  }


\cventry
  {Computational Research Assistant (Internship)} % Job title
  {CANMET Materials Technology Laboratory (NRCan)} % Organization
  {Ottawa, ON, Canada} % Location
  {2008 - 2009} % Date(s)
  {
    \begin{itemize} % Description(s) of tasks/responsibilities
      \liststyle
      \item {a}
    \end{itemize}
  }
\mynotes{
}



\cventry
  {Research Assistant (Internship)} % Job title
  {Center of Atmospheric Sciences (University of Waterloo)} % Organization
  {Waterloo, ON, Canada} % Date(s)
  {Jan 2008 - Apr 2008} % Location
  {
    \begin{itemize} % Description(s) of tasks/responsibilities
      \liststyle
      \item {helped reproduce snowflake-like ice particles with clear crystalline shapes for the first time in the lab's flowtube. The key was adding more moisture.}
      \item {\tgproject{3D shape identification} designed and implemented a program to identify the 3D shape of ice particles from images taken in the flowtube. 
      The program applied the theory of discrete Fourier shape descriptors. 
      I reduced a core integral equation used at the lab into a simpler form.
      \tgskills: \rfcpp, \rfmysql}
    \end{itemize}
  }
\mynotes{
- helped reproduce snowflake-like ice particles with clear crystalline shapes for the first time in the lab's flowtube.
The key was adding more moisture.

- 3D shape identification of ice particles:
Ice particle 3D shape identification program
Method: Fourier shape descriptor 
i.e. a set of numbers associated with a given set of coordinates. 
Represents shape since it has the same invariants as a shape i.e. invariant under rotation, scaling

Implementation: C/C++

Successfully improved accuracy of ice particle shape identification using this program
Program scans hundreds of images taken in the lab efficiently and categorized each.

Skills: C/C++, MySQL
}

\cventry
  {Quality Analyst (Internship)} % Job title
  {Quest Software} % Organization
  {Toronto, ON, Canada} % Location
  {May 2007 - Aug 2007} % Date(s)
  {
    \begin{itemize} % Description(s) of tasks/responsibilities
      \liststyle
      \item {Tested Foglight, a performance monitoring software, on various systems including openSUSE, Red Hat Linux, Solaris, and HP-UX.}
    \end{itemize}
  }
\mynotes{
Automated setups and tests using shell scripts.
}

\cventry
  {IT Assistant (Internship)} % Job title
  {Canadian Academy} % Organization
  {Kobe, Hyogo, Japan} % Date(s)
  {Sep 2006 - Dec 2006} % Location
  {
    \begin{itemize} % Description(s) of tasks/responsibilities
      \liststyle
      \item {Provided technical assistance to school staff.}
      \item {\tgproject{file administration} deployed \href{http://www.radmind.org/}{Radmind}, an open source remote file system administration software for synchronizing managed files across all school lab computers.}
    \end{itemize}
  }
\mynotes{
- Provided technical assistance to school staff
- Deployed Radmind, an open source file administration software for synchronizing managed files across all school lab computers.
Conducted workshop to school tech team on usage
}
